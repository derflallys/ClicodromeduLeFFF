\documentclass[12pt,a4paper]{article}

\usepackage{fullpage}
\usepackage[utf8]{inputenc}
\usepackage{amsfonts}
\usepackage{amssymb}
\usepackage[french]{babel}
\usepackage[cyr]{aeguill}
\usepackage{natbib}
\usepackage{graphicx}
\usepackage{tabularx}
\usepackage{hyperref}
\usepackage[document]{ragged2e}
\setlength{\parindent}{2em}
\setlength{\parskip}{1em}

\newcommand{\quotes}[1]{``#1''}

\begin{document}

\begin{titlepage}
\centering
{\scshape\LARGE Université de Bordeaux \par}
{\scshape\Large Master 1 Informatique  \par}
\vspace{3cm}

{\Huge\bfseries Projet de Programmation\par}
{\Huge\bfseries Le Clicodrome de LEFFF \par}
{\Large\bfseries par Lionel CLEMENT \par}
\vspace{0.5cm}
{\Large\itshape Cahier des besoins\par}
{\large 01 février 2019\par}

\vfill
réalisé par \par
BAKIR \textsc{Fatima Ezzahra} \par
JELLITE \textsc{Oumayma} \par
NEDELEC \textsc{Guillaume} \par
SYLLA  \textsc{Alfred Aboubacar} \par
\vfill

{\large Enseignants responsables : Philippe NARBEL et Vincent PENELLE\par}

\end{titlepage}

\newpage
\tableofcontents

\newpage\section{Introduction}

\subsection{Présentation du projet}

\smallbreak Lionel CLEMENT est un enseignant-chercheur du Labri (Laboratoire Bordelais de Recherche en Informatique) informatique, spécialisé dans le domaine de la linguistique formelle et du traitement automatiques des langues. Il a réalisé avec Benoît SAGOT, chercheur dans le même domaine à l'Inria (Institut National de Recherche en Informatique et en Automatique), Le Lexique des formes fléchies du français, appelé le LeFFF.

Avant de détailler l’objectif de notre projet, nous allons mettre
au clair quelques notions nécessaires à la compréhension du sujet. 
Il faut savoir que dans la langue française, les formes fléchies d'un mot correspondent à sa conjugaison et à ses déclinaisons (sa forme au pluriel, les mots ayant une correspondance comme "mien" et "moi" par exemple).
De plus ce lexique est un “dictionnaire” contenant un grands nombre de mots de différentes catégories. Ce sont des noms propres, des noms communs, des adverbes, des adjectifs, des tous les autres types de mots possible dans la langue française.
Notre but sera de trier tous ces types de mots afin de déterminer quelles sont les formes fléchies et quelles sont les mots "racines".

Lionel CLEMENT qualifie ce lexique comme "une ressource complexe constituée de
\begin{itemize}
\item Lexème ou grammème (ie Prendre)
\item Vocables (ie Prendre=saisir, Prendre=recevoir)
\item Catégories syntaxiques (ie Verbe)
\item Sous-catégories syntaxiques (ie Transitif, passivable)
\item Catégories grammaticales (ie Nombre$\rightarrow\{sing, plur\}$, Personne$\rightarrow\{1, 2, 3\}$)
\item Règles de flexion (ie Table de conjugaison de prendre - Stem=.*(pren|mett))
\item Valence (ie objet nominal, oblique en à)
\item Réalisation syntaxique (ie passif en par)
\item Phraséologie (ie "Prendre ombrage", "Prendre ses jambes à son cou")
\item Collocation (ie "Prendre une initiative")
\item Fonctions lexicales (ie Magn: "Prendre une belle initiative")
\end{itemize}

\smallbreak Aujourd'hui ce lexique est présent dans un fichier texte et il n'existe pas d'outils permettant de faciliter son accès, sa modification et son enrichissement.
Le système actuel n'assure pas une bonne traçabilité des modifications des mots effectués par les linguistes et ne permet aucun contrôle de ces modifications. 

L'objectif de ce projet est donc de transformer le lexique au format texte en une base de données et de développer une interface web facilitant les interactions avec ce lexique via cette base de données.
Le projet peut être donc séparé en trois parties principales : 
\begin{itemize}
    \item L'importation du lexique au format texte dans une base de données
    \item La création de l'interface web qui interagira avec la base de données
    \item Le développement d'un compilateur permettant de générer les formes fléchies d"un mot donné en entrée
\end{itemize}


\newpage
\subsection{Description de l'existant et choix de conception}

Pour ce projet, nous avons a notre  disposition différentes versions  du Lefff  dont la dernière réalisé par Benoît Sagot en 2010. Cette version regroupe diffèrent format du lefff , à savoir \cite{lefff_int} :
\begin{itemize}
\item le lexique \textbf{intentionnel}, qui décrit pour chaque entrée son lemme (forme canonique + tableau d'inflexion) ainsi que des informations syntaxiques profondes (cadre de sous-catégorisation profonde + réalisations possibles + restructurations possibles) .
\item le lexique \textbf{extensionnel}, construit automatiquement par compilation du lexique intentionnel. Ce processus de génération comprend une étape d'inflexion, en fonction de la classe d'inflexion associée à l'entrée intentionnelle, puis une étape de construction des différentes structures syntaxiques (une pour chaque restructuration pertinente) associées à chaque forme infléchie (les informations syntaxiques peuvent varier d'une forme à une autre, notamment les formes infinitive, de participant, d'une restructuration à une autre).
\end{itemize} 

\smallbreak Les formats de la dernière version de lefff ont des extensions et contenues différents, le lexique \textbf{intentionnel} est manipulable grâce à l'outil "alexina-tools" qui permet de le compiler pour avoir le format extensionnel avec seulement les paramètres morphologiques et une avec tous les paramètres d'un mot .

Le lexique extensionnel avec seulement les paramètres morphologiques est présenté sous la forme ci-dessous :
\begin{verbatim}
démariés	adj	démarier	Kmp
démariés	v	démarier	Kmp
démarqua	v	démarquer	J3s
démarquage	nc	démarquage	ms
démarquages	nc	démarquage	mp
\end{verbatim}

\smallbreak La première colonne comprend le mot recherché. 
\smallbreak La seconde présente sa catégorie (\textbf{adj}(adjectif), \textbf{v}(verbe), \textbf{np}(nom propre), \textbf{nc} (nom commun).
\smallbreak La troisième colonne présente le mot auxquel est lié le mot de la première colonne.
Si la première colonne et la troisième sont différentes, cela signifie que le mot de la première colonne est une forme fléchie du mot de la troisième colonne. S'il les 2 colonnes sont identiques alors c'est que ce mot est un mot que l'on sauvegardera dans la base de données avec ses attributs.
Enfin la dernière colonne apporte des informations sur le mot comme le genre, s'il est singulier pluriel ou encore a quel personne le verbe est employé. Une explication de ces codes est disponible sur le site de Lionel CLEMENT \cite{tagset}.

La différence entre le contenu des différents formats de fichier du LeFFF à notre disposition est principalement le détail par rapport au paramètre morphologique du mot.
Lionel CLEMENT nous a aussi parlé d'une implémentation de PFM (Paradigm functions for periphrasis) qui permet de générer les formes fléchies d'un mot en prenant comme entrée les paramètres morphologique et le lexème (une unité minimale de sens dans une langue).


\section{Besoins fonctionnels}
\smallbreak Lors de nos communication avec le client, nous avons pu identifié ses principaux besoins pour le projet. En effet, le projet se découpe en trois partie principales : 
\begin{itemize}  
  \item Une application web permettant aux utilisateurs d'interagir facilement avec le lexique
  \item une base de données contenant les différents mot du lexique (les formes fléchies des mots sont exclus de la base) et leurs attributs.
  \item Un compilateur permettant, pour un mot donné, de générer ses formes fléchies.
\end{itemize}

\smallbreak Pour apporter un contrôle des interactions des utilisateurs depuis le site web, un système de rôle sur le site sera mis en place. Avant de présenter la liste des besoins fonctionnels identifiés, voici une présentation de ces différents rôles, qui vous aidera a mieux comprendre ces besoins : 
\begin{itemize}  
  \item \textbf{Les visiteurs} : Ils ne peuvent que rechercher, consulter les mots du lexique, signaler des erreurs sur un mot ou exporter le lexique.
  \item \textbf{Les éditeurs} : Ce type d'utilisateur peut, en plus d'avoir les mêmes droits que les visiteurs, modifier (le mot ou ses attributs) un mot du lexique. Ces actions ne peuvent être effectué qu'en étant connecté (système d'authentification) sur le site.
  \item \textbf{Les modérateurs} :  Ces utilisateurs sont des éditeurs pouvant supprimer des mots du LeFFF et qui ont un accès à la liste des utilisateurs ayant un compte sur le site.
  \item \textbf{Les administrateurs} : Ils ont un contrôle total de l'application. En plus d'agir comme des modérateurs, c'est eux qui valident les inscriptions sur site, gèrent les rôle des différents utilisateurs et ont le pouvoir de bannir des utilisateurs du site.
\end{itemize}
 
\subsubsection{Importer le LeFFF dans une base de données relationnelle}
\textbf{Priorité : 1} \\ 
Cette fonctionnalité permet d'importer les données du lexique dans un format donné (.txt / .mlex) directement dans une base de données où l'architecture sera préalablement configuré.
Cette fonctionnalité est nécessaire pour alimenter notre base de donnée puisque le leFFF comprend un très grand nombre de données (environ 500 000 lignes).
Pour l'implémenter, nous allons définir un algorithme qui permet de découper le fichier d'entrée, de filtrer les mots a enregistré et d'effectuer des requêtes d'ajout dans notre base de données

 \subsubsection{Génération des formes fléchies}
\textbf{Priorité : 1} \\
Le compilateur nous permet de générer les formes fléchies d'un mot donné en entrée. Nous somme toujours à l'étude du comportement de se compilateur pour qu'il arrive à effectuer son rôle correctement.

\subsubsection{Rechercher un mot}
 \textbf{Priorité : 2}
\\ Nous allons mettre à la disposition de tous les utilisateurs un champ de saisie sur toutes les pages de l'application web leur permettant de rechercher un mot. A la manière d'un moteur de recherche classique, la liste des mots enregistrés dans notre base de données, correspondant plus ou moins à ce qui a été recherché apparaîtront sur la page. Il sera ensuite possible à l'utilisateur de consulter en détail un mot, de signaler une erreur, de le modifier ou le supprimer (Si les droits de l'utilisateur le lui permet). Si aucun mot ne correspond en base de données, un message d'avertissement apparaîtra sur l'interface.
\begin{center}\includegraphics[width=150mm]{img/search_screen.png}\end{center}

\subsubsection{Consulter un mot}
 \textbf{Priorité : 2}
\\ Tous les utilisateurs du site web peuvent consulter un mot. Pour cela il suffit à l'utilisateur de rechercher un mot (voir la partie "Rechercher un mot") et de sélectionné le mot voulu. Ce mot sera ensuite envoyé à notre compilateur pour générer ses formes fléchies et les renvoyer à notre application afin d'avoir toutes les informations du mot affichées sur l'interface.
Si les droits de l'utilisateur le permet, les boutons d'ajout, de modification et de suppression apparaissent sur cet écran.
\begin{center}\includegraphics[width=150mm]{img/consult_screen.png}\end{center}

\subsubsection{Ajouter un mot}
 \textbf{Priorité : 2}
\\ Cette fonctionnalité est réservé au utilisateurs authentifiés (les administrateur ,les éditeurs et les modérateurs). Un formulaire est proposé, où il est demandé d'ajouter le mot, son type (adjectifs, nom commun etc...), sa nature (féminin ou masculin), s'il est singulier ou pluriel. Les champs s'adaptent selon le type sélectionné. Par exemple, si le type "verbe" est sélectionné, alors des champs concernant le temps et la personne seront ajouté.
Avant d'ajouter le mot en base, des vérifications seront effectués afin  de vérifier si le mot n'existe pas déjà en base, ou si le mot n'est pas une forme fléchies d'un mot en base de données.
\begin{center}\includegraphics[width=150mm]{img/add_modif_screen.png}\end{center}

\subsubsection{Supprimer un mot}
 \textbf{Priorité : 2} \\ 
 Les modérateurs et les administrateurs peuvent supprimer un mot de la base de données. Il suffit simplement de rechercher le mot et d'appuyer sur "Supprimer". Une notification sera envoyé aux administrateurs pour les informer de la suppression du mot. Le mot sera alors supprimé de la base de données.

\subsubsection{Modifier un mot}
 \textbf{Priorité : 2} \\ 
 La fonction de modification d'un mot ressemble à celle d'ajout (voir partie Ajouter un mot). Elle est accessible a tous les utilisateurs connectés. En effet le formulaire sera le même mais pré rempli avec les informations existante du mot. Il suffit à  l'utilisateur d'effectuer les modification qu'il souhaite et de les valider pour que les changements soient persisté en base de données.

\subsubsection{Signaler un mot}
 \textbf{Priorité : 2}
 \\ Les visiteurs simple et les éditeurs ont la possibilité de signaler des erreurs sur un mot. Ce mécanisme permet d'envoyer une notification aux administrateurs pour leur faire part d'informations sur un mot en particulier. 

\subsubsection{Contrôler les interactions }
 \textbf{Priorité : 2}
\\ Tous les utilisateurs doivent s'authentifier pour pouvoir modifier où enrichir le LeFFF. Afin d'ajouter de la sécurité, nous avons décider de mettre constamment des message de validation pour être sur que l'utilisateur souhaite réellement effectuer une action et ne pas faire d'erreur.
De plus un système de déconnexion automatique après une période d'inactivité (environ 10 minutes) sera mis en place en cas d'oubli de fermeture de session afin que personne ne puisse subtiliser la session.
Enfin des logs seront implémentés afin de tracer les changements effectués en plus de l'historique.

\subsubsection{Gérer les rôles:}
 \textbf{Priorité : 3} \\
Pour qu'un utilisateur puisse effectuer des tâches d'ajout, de modification ou de suppression de mot, il faut qu'un administrateur lui affecte ces droits (les rôle d'éditeur ou modérateur). A cet égard, on doit créer une interface où l'administrateur puisse faire cela.
Sur cette interface, apparaîtra la liste des comptes utilisateurs, pour chaque utilisateur se trouvera des boutons  : \textbf{Supprimer} cet utilisateur ,\textbf{ Modifier} le rôle de cet utilisateur. 
Sur la même interface on va mettre en place un bouton pour \textbf{Valider} l'inscription d'un nouvel utilisateur.
\begin{center}\includegraphics[width=150mm]{img/user_screen.png}\end{center}

 \subsubsection{Exporter le Lefff}
\textbf{Priorité : 3}  \\
Tous les utilisateurs peuvent effectuer une exportation du lexique. Cette option sera disponible depuis notre application web. Cette opération nécessite l'utilisation du compilateur PFM pour générer les formes fléchis de tous les mots présents dans la base de données.
Nous proposerons aussi une exportation du lexique sans les formes fléchies, qui sera bien plus rapide et légère.


\subsubsection{Gestion de l'historique d'un mot}
 \textbf{Priorité : 3} \\
Nous avons décidé de mettre en place un historique permettant d'enregistrer tous les ajouts, modifications et suppression. Pour cela nous voulons implémenter de l'Event Sourcing permettant d'enregistrer tous les évènements exécutés plutôt que d'enregistrer les données modifiées en base de données.


\section{Les besoins non fonctionnels}
\smallbreak

\subsubsection{Sécurité}
\textbf{Priorité : 1}
\smallbreak
Afin de protéger l'intégrité des données du LeFFF dans la base de données, nous sécuriserons la base et l'application web des problèmes de sécurité suivants :
\begin{itemize}
    \item \textbf{Les injections SQL}, permettant de récupérer ou d'altérer des données, voir même prendre le contrôle de serveur. L'attaque par injection SQL consiste à injecter du code malicieux SQL qui sera interprété par le moteur de base de données dans un champ de saisie d'une application web. \\
    \textbf{Notre solution :} Afin de se protéger de se genre d'attaque, nous avons choisi de développer l'application web en PHP avec le framework Symfony, qui comprend un l'ORM (Object-Relational Mapping) Doctrine. Cet ORM est une couche d'abstraction de base de données qui permet d'interagir avec le contenu de la base de données en manipulant des objets (informatique) du coté de notre application. Il comprend des modules de sécurité (génération de contraintes, formatage des données envoyées ect...) qui détecte les attaques par injection SQL.
    
    \item \textbf{Le Cross-Site Scripting (XSS)} : Cette faille permet d'injecter du code malicieux sur une page web. Par exemple sur un forum, si quelqu'un entre en commentaire un code Javascript nuisible, tous les autres utilisateurs qui accéderont à cette page seront impacté par ce code qui s'exécutera sur leur navigateur. \\
    \textbf{Notre solution :}Pour remédier à cela, le framework Symfony possède aussi un système (filtrage des données d'entrées, échappement des données en sortie...) protégeant de ce genre d'attaque afin de filtrer tous ce qui pourrait ressembler à du code. C'est un procédé assez similaire à celui de la protection contre les injections SQL.
    
    \item \textbf{La falsification de requête inter-sites (CSRF)} : Cette faille permet à un attaquant de forcer ses victimes à effectuer certaines actions sur un site cible, sans qu’elles s’en aperçoivent. Pour cela, les attaquants cherche à ce que leur victime visite une page (en étant connecté) où des scripts malicieux seront exécutes sans que la victime s'en aperçoive à l'aide de balise image chargeant le script par exemple. La victime étant authentifié sur le site, les scripts pourront s'exécuter correctement et corrompre les données du site. \\
    \textbf{Notre solution :} Voici la dernière faille de sécurité à laquelle Symfony nous protège. En effet le framework fait cela en générèrant des jetons ("tokens") au moment de l'authentification, différents pour chaque utilisateur et qui sont stockés dans la session. Ce jeton est ensuite vérifié a chaque action avec la base afin de vérifier que c'est bien une action voulue par l'utilisateur et non une attaque CSRF. 
    
    \item \textbf{Le déchiffrement de données} qui consiste à déchiffrer des données crypté. Cela peut permettre de trouver des mot de passe et donc prendre le contrôle d'une session utilisateur ou bien de découvrir des informations sensibles (informations bancaire, personnelles...) \\
    \textbf{Notre solution :} Pour éviter cela, nous avons décidé de chiffrer les données avec des algorithmes irréversible (une fois crypté, aucune technique connue à ce jour peut décrypter le mot, il faut forcement connaître le mot initial) comme "bcrypt" ou "sha512".
    
    \item \textbf{L'écoute de communication réseau} : Requêter un site web avec un protocole HTTP permet à des intrus présent sur votre réseau de lire ou modifier le site internet que vous souhaitez consultez et présente donc un gros risque. \\
    \textbf{Notre solution} : Utiliser le protocole HTTPS qui permet de chiffrer les données échangées et ainsi les protéger de toutes interceptions ou modification.
\end{itemize}

\subsubsection{Format de fichier}
\textbf{Priorité : 2}
\smallbreak
\begin{itemize}
\item Afin de pouvoir répondre aux besoins d'importation et d'exportation du LeFFF avec la base de données, il nous faut définir les format de fichiers compatibles avec ces fonctions.
Nous avons fait le choix d'utiliser les formats textes et elex (.txt et .elex)  car ce sont les formats disponibles au téléchargement du LeFFF actuellement. Ce choix nous permet de pouvoir exploiter le LeFFF existant sans avoir à le reformatter dans un autre format.
\end{itemize}

\subsubsection{Performances}
\textbf{Priorité : 3}
\smallbreak
Au niveau des performances, nous avons décider de faire une implémentation du projet de manière à ce que la requête de demande des formes fléchies d'un mot (requête en base et génération des formes fléchies par le compilateur) n'excède pas les deux secondes d'exécutions. De plus les actions d'importation et d'exportation étant plus lourdes, nous voulons implémenter ces fonctions pour qu'elles n'excèdent pas 10 minutes d'exécution.

\subsubsection{Ergonomie}
\textbf{Priorité : 4}
\smallbreak
Nous avons fait le choix d'adopter une design très simple et épurée afin que l'utilisateur ne se perde pas dans l'interface et puisse trouver des solutions à son besoin de manière simple et naturelle.
\smallbreak
Des codes couleurs et icônes distinctifs seront mis en place afin que l'utilisateur comprennent le plus rapidement possible à quoi il est confronté. (par exemple, les messages en rouge correspondront à des messages d'erreurs, le vert pour la validation et du jaune pour des avertissements.).

\section{Planning du projet}
Pour ce planning prévisionnel, nous nous réferrons aux semaines indiquées par M. NARBEL à l'adresse suivante : 
\smallbreak
http://dept-info.labri.fr/~narbel/PdP/calendar19.html

\begin{tabularx}{\textwidth}{|p{7cm}|c|X|}
    \hline
    \textbf{Tâches} & 
    \textbf{Semaines} & 
    \textbf{Acteurs} \tabularnewline
    \hline
    Rédaction du cahier des besoins & 1 à 4 & TOUS \tabularnewline
    \hline
    Mise en place de l'architecture de la base de données  & 5 à 6  & TOUS \tabularnewline
    \hline
    Importation du LeFFF dans la base de données  & 5 à 6 & Alfred et Guillaume \tabularnewline
    \hline
    Conception du compilateur & 5 à 6 & Oumayma et Fatima Ezzahra  \tabularnewline
    \hline
    Conception de l'application web & 7 à 8 & Fatima Ezzahra et Alfred  \tabularnewline
    \hline
    Développement du compilateur & 8 à 12 & Alfred et Oumayma \tabularnewline
    \hline
    Développement de l'application web & 8 à 12 & Fatima Ezzahra et Guillaume \tabularnewline
    \hline
    Interfaçage entre l'application web, le compilateur et la base de données & 12 & TOUS \tabularnewline
    \hline
    Phases de tests unitaires & 12 à 13 & TOUS \tabularnewline
   \hline
    Soutenance & 14 & TOUS \tabularnewline
   \hline
\end{tabularx}

\section{Outils de développement}

Ce projet peut être divisé en 3 parties :
\begin{itemize}  
  \item l'application web
  \item le compilateur 
  \item la base de données
\end{itemize}

\subsection{Application web : PHP avec le framework Symfony 4}
Nous avons choisi de développer le site web en PHP car c'est un langage simple d'utilisation et que ce framework apporte beaucoup d'avantages en termes de sécurité (comme énoncé précédemment dans la partie 2.2.1) et de développement (configuration simple et rapide ect...)

\smallbreak
\subsection{Compilateur PFM : Java}
Pour le compilateur, nous avons choisi d'utiliser le langage Java car il existe de nombreuses nombreuses API (Application Programming Interface) publiques pouvant aidés à notre développement. 
De plus, des librairies pour la manipulation de PFM sont disponibles pour le Java, un élément nécessaire pour ce projet.

\subsection{Base de données : MySQL}
Pour la base de données, le choix s'est porté sur MySQL pour sa facilité d'utilisation et sa bonne compatibilité avec le framework Symfony utilisé pour le site web. 

\section{Références}
\bibliography{references}
\bibliographystyle{plain}
\end{document}