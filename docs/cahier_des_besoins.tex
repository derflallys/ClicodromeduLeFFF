\documentclass[12pt,a4paper]{article}

\usepackage{fullpage}
\usepackage[utf8]{inputenc}
\usepackage{amsfonts}
\usepackage{amssymb}
\usepackage[french]{babel}
\usepackage[cyr]{aeguill}
\usepackage{natbib}
\usepackage{graphicx}
\usepackage{tabularx}
\usepackage{hyperref}
\usepackage[document]{ragged2e}
\setlength{\parindent}{2em}
\setlength{\parskip}{1em}

\newcommand{\quotes}[1]{``#1''}

\begin{document}

\begin{titlepage}
\centering
{\scshape\LARGE Université de Bordeaux \par}
{\scshape\Large Master 1 Informatique  \par}
\vspace{3cm}

{\Huge\bfseries Projet de Programmation\par}
{\Huge\bfseries Le Clicodrome de LEFFF \par}
\vspace{0.5cm}
{\Large\itshape Cahier des besoins\par}
{\large 01 février 2019\par}

\vfill
réalisé par \par
BAKIR \textsc{Fatima Ezzahra} \par
JELLITE \textsc{Oumayma} \par
NEDELEC \textsc{Guillaume} \par
SYLLA  \textsc{Alfred Aboubacar} \par
\vfill

{\large Enseignants responsables : Philippe NARBEL et Vincent PENELLE\par}

\end{titlepage}

\tableofcontents

\newpage\section{Introduction}

\subsection{Présentation du projet}


\smallbreak
le Lexique des formes fléchies du français ; LEFFF est un lexique de la langue française mis en place par notre client Lionnel Clement et Benoit Sagot.Ce lexique est obtenu par des nouvelles techniques qui repose sur le fait d'associer a chaque forme fléchis des informations morphologiques et de factoriser l'information en utilisant une structure d'héritage.Ce n'est donc pas une simple liste de mots ,mais une ressource complexe constituée de Lexème,vocables,catégories syntaxiques,etc

 On propose donc de fournir un site web interactif qui permet d'interagir avec le lexique pour enrichir les données morphologiques et syntaxiques et d'obtenir une extraction du lexique .Ce dernier va être enregistré sous la forme d'une base de données relationnelle qui va servir dans un premier temps a éditer le lexique par exemple ajouter ou supprimer un mot . Ensuite pour consulter ou exporter le lexique on va utiliser le \textbf{PFM} ,c'est est transducteur qui permet par composition de construire les formes fléchis .

 Pour la bonne compréhension de notre projet nous allons clarifier quelques notions. D'abord une forme fléchis est un modèle qui s'emploie pour indiquer la liste des modes,temps d'un verbe français conjugue et personnes pour lequel cette forme fléchie est utilisée ,en plus elle s'emploie pas pour l'infinitif .
\\ Exemple :
\textbf{{{fr-verbe-flexion|payer|ind.p.2s=oui|sub.p.2s=oui|imp.p.2s.postposé=oui}} payes,  \textbf{forme fléchie} du verbe paye }
\\  \textit{ ind.p.2s=oui} :  2e personne du singulier, impératif présent
\\  \textit{ imp.p.2s.postposé=oui} :   2e personne du singulier, impératif présent, cas particulier devant -en et -y, verbes du premier groupe et aller.




\newpage
\subsection{Description de l'existant et choix de conception}

Pour ce projet, nous avons a notre de disposition le lexique LeFFF mis à jour en avril 2006 au format texte (.txt).
Ce lexique contient un ensemble de forme flechies de mots français présentés de la manière suivante :
\begin{verbatim}
  débouler	v	[pred='débouler_____1<suj:(sn|sinf|scompl),obj:(sn)>',cat=v,@W]
  à bon marché	adv	[pred='à bon marché_____1',cat=adv,advGP=+]
  à bientôt	  pres	[pred='à bientôt_____1',advGP=+]
  Égypte		np	[pred='Égypte_____1<suj:(sn)>',@loc,@fs]
\end{verbatim}

EXPLIQUER LA SYNTAXE DU LEFFF (EX CODE CI DESSUS)

Ce lexique est dsponible sous licence « LGPLLR » (Lesser General Public License For Linguistic Resources) à l'adresse suivante : 
\\ \begin{center}http://www.labri.fr/perso/clement/lefff/telechargement.html\end{center}

  \section{Analyse des besoins}

\subsection{Les besoins fonctionnels:}

Sur le site internet, il existe 4 types d’utilisateurs : Les Administrateurs (A) , les Éditeurs ou linguistes (E), les Modérateurs (M) et les Visiteurs (V).
Chacun de ses utilisateurs possède des accès particuliers sur le site internet.
Les visiteurs ne possèdent pas de compte sur la plate-forme et ne peuvent donc pas s'authentifier . 

\subsubsection{Le site web interactif}
\begin{tabularx}{\textwidth}{|l|c|X|}
  \hline
  \textbf{Besoins} & 
  \textbf{Acteurs} & 
  \textbf{Explication} \\
  \hline
  Authentifier les utilisateurs & 
  A, E, M & 
  Chaque utilisateur ayant un compte sur le site peut s'authentifier pour avoir accès a l'interface d'administration \\ 
  \hline
  Gérer les rôles & A
   & Permet d'assigner les ôles (A,E,M) aux différents utilisateurs ayant un compte sur le site.
 \\
  \hline
  Bannir un utilisateur& 
  A & 
  L'administrateur peut bannir tous les autres utilisateurs de la plateforme \\
  \hline
  Rechercher un mot & 
  A , E, M, V & 
  Rechercher un mot dans le lexique pour accéder à ses données \\
  \hline
  Consulter un mot &
  A, E, M, V &
  Permet de consulter les données d'un mot du lexique \\
  \hline
  Modifier un mot &
  A, E, M & 
  Permet de modifier les données liées à un mot \\
  \hline
  Supprimer un mot &
  A, E, M & 
  Permet de supprimer un mot du lexique \\
  \hline
  Faire un signalement & 
  V &
  Permet de signaler une erreur et de proposer une éventuelle correction \\
  \hline
  Valider inscription Éditeur &
  A &
  Après l'inscription d'un éditeur sur le site, l'administrateur doit activer son compte après vérification \\
  \hline
  Signaler un mot &
  V, E &
  Un visiteur ou un éditeur peut signaler un mot qui comporte des erreurs ou qui est fausse \\
  \hline
  Exporter le lexique &
  V, E, A , M &
  Les utilisateurs du site peuvent exporter LeFFF du site sous différents formats de fichier (a définir)\\
  \hline
\end{tabularx}
\subsubsection{Le diagramme des cas d'utilisation:}
  \includegraphics[width=18cm]{img/Diagram_UseCase.PNG}

\subsection{Les besoins non fonctionnels}

\subsubsection{Sécurité}
\begin{itemize}
 \item Les données du LeFFF sont protégée. En effet toute intéraction avec ces données (modification / ajout ou suppression) ne peut être effectuée que par des utilisateurs authentifié et par conséquent, validé par unn administrateur.
 \item Le système de rôle (Visiteur, Editeur, Modérateur et Administrateur) permet de réguler les droits de chacun.
 \item Une déconnexion automatique après une période d'inactivité (durée à définir) sera mis en place.
 \item Des logs seront implémentés afin de tracer les changements effectués.
 \item Un historique par rapport à chaque mot du lexique sera mis en place afin de connaitre la date de dernière modification ainsi que l'utilisateur acteur de cette modification.
\end{itemize}
\subsubsection{Ergonomie}
\begin{itemize}
\item Nous avons fait le choix d'adpoter une design très simple et épurée afin que l'utilisateur ne se perde pas dans l'interface et puisse trouver des solutions à son besoin de maninère simple et naturelle.
\item Des boites de dialogue et des messages de confirmation seront affichés à chaque opération effectuée par l'utilisateur afin d'éviter les erreurs dans les actions utilisateurs.
\item Des codes couleurs et icones distinctifs seront mis en place afin que l'utilsiateur comprennent le plus rapidement possible à quoi il est confronté. (par exemple, les messages en rouge correspondront à des messages d'erreurs, le vert pour la validation et du jaune pour des avertissements.).
\end{itemize}

\section{Planning du projet}

\begin{tabularx}{\textwidth}{|l|c|X|}
  \hline
  \textbf{Tâche } & 
  \textbf{Date de début} & 
  \textbf{Date de fin} \\
  \hline
  \textbf{Capture des besoins fonctionnels} :& 18/01/2019 & 07/02/2019
  \\ 
     \hline
 
 Étude de projet : & 18/01/2019 &  07/02/2019
    \\
  \hline
 Recherche sur Internet sur les outils à utiliser  & 28/01/2019 &  03/02/2019
    \\
  \hline
 Définition des besoins et installation des outils  & 04/02/2019 &  05/02/2019
    \\
  \hline
  \textbf{Analyse et conception} :& 28/01/2019 & 01/02/2019
  \\ 
   \hline
  Les diagrammes des cas d'utilisation & 28/01/2019 &  01/02/2019 
    \\
  \hline
  Diagrammes de séquence & 31/01/2019 &  01/02/2019
    \\
    \hline
\textbf{Codage et test}: & 04/02/2019 &  03/04/2019
    \\
      \hline
La création de la base de données & 04/02/2019 &  05/02/2019
    \\
     \hline
Implémentation du compilateur & 07/02/2019 &  03/04/2019
    \\
         \hline
la création des interfaces & 15/03/2019 &  02/04/2019
    \\
       \hline
       

\textbf{Rédaction du Rapport}: & 05/03/2019 &  05/04/2019
  \\
  \hline

\end{tabularx}

\section{Outils de développement}

Ce projet peut être divisé en 3 parties :
\begin{itemize}  
  \item le site web
  \item le compilateur 
  \item la base de données
\end{itemize}

\subsection{Back-end du site web : PHP avec le framework Symfony 4}
\begin{center}
  \includegraphics[width=2cm]{img/php.png}
  \includegraphics[width=2cm]{img/symfony.png}
\end{center}

Nous avons choisi de développer le site web en PHP car c'est un langage permettant de faire de la programmation orienté objet. 
De plus c'est un langage très performant et simple à mettre en place.
Nous avons décidé d'utiliser le framework Symfony car un framework a beaucoup d'avantages comme :
\begin{itemize}  
  \item une architecture MVC (Modèle-Vue-Contrôleur) permettant de découper le code en suivant une logique métier.
  \item la facilité de création de tests automatisés
  \item une sécurisation de l'application web (XSS, CSRF et injection SQL)
  \item Un ensemble de packages déjà développés à intégrer à l'application.
  \item Des performances optimisés avec une utilisation du cache optimisée
  \item Une communauté importante et beaucoup de documentation
\end{itemize}

\smallbreak

Nous utiliserons ce framework dans sa version 4.1 car c'est une version très légère par rapport à la version 3. 
En effet aucun packages complémentaires au noyau n'est installé. De plus le déploiement et la configuration du framework a été très simplifiée dans cette version.
Cette version de Symfony nécessite une version de PHP supérieure ou égale à 7.1.3

\subsection{Front-end du site web : Framework Javascript Angular}
\begin{center}
  \includegraphics[width=2cm]{img/angular.png}
\end{center}
On a décidé d'utiliser un framework Javascript pour la partie front-end du site.
Les framework javascript permettent la création d'interface dynamique tout en gardant un code très structuré, ce qui facilite grandement la maintenance de l'application par la suite.
L'utilisation de Angular avec TypeScript permet de typer le javascript et d'adoper le nouveau standard Javascript ES6.

\subsection{Compilateur PFM : Java}
\begin{center}
  \includegraphics[width=2cm]{img/java.png}
\end{center}
On a choisi d'utiliser le langage Java pour le compilateur PFM car c'est un langage orienté objet. 
Cette caractéristique permet la mise en place du DDD (Domain Driven Design) qui va permettre une structure de l'application autour du domaine.
Ce langage dispose de nombreuses API (Application Programming Interface) publiques pouvant être utilisées sans avoir a réimplémenter les fonctionnalités de ces API. 
De plus, des librairies pour la manipulation de PFM sont disponibles pour le Java, un élément nécessaire pour ce projet.

\subsection{Base de données : MySQL}
\begin{center}
  \includegraphics[width=2cm]{img/mysql.png}
\end{center}
Pour la base de données, le choix s'est porté sur MySQL pour sa facilité d'utilisation et ses bonnes performances. 
De plus, ce Système de Gestion de Base de Données (SGBD) s'intègre facilement à un environnement PHP tel que celui qu'on le souhaite utiliser.


\section{Questionnement}
\begin{itemize}  
  \item que fait réellement le compilateur ?
  \item Quand doit il être utiliser ?
  \item Que prend il en entrée ?
  \item Quel type de données renvoie t -il?
  \item Le compilateur est il nécessaire si tout est enregistré en base ?
  \item Que signifie le format de données du LeFFF (voir dans l'intro) ?
  \item Comment lier les formes fléchies (pour configurer la base) ?
  \item Au niveau des rôle, qui peut supprimer des mots ? Dofférence entre modérateur et editeur ?
  \item 
\end{itemize}

\bibliography{references}
\bibliographystyle{plain}
<<<<<<< HEAD
\end{document}
=======
\end{document}
>>>>>>> 8cda4562cfd73d41eb8cc1411209dfc8fea34981
