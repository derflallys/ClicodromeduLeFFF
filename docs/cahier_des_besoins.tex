\documentclass[12pt,a4paper]{article}

\usepackage{fullpage}
\usepackage[utf8]{inputenc}
\usepackage{amsfonts}
\usepackage{amssymb}
\usepackage[french]{babel}
\usepackage[cyr]{aeguill}
\usepackage{natbib}
\usepackage{graphicx}
\usepackage{tabularx}
\usepackage[document]{ragged2e}
\setlength{\parindent}{2em}
\setlength{\parskip}{1em}

\newcommand{\quotes}[1]{``#1''}

\begin{document}

\begin{titlepage}
\centering
{\scshape\LARGE Université de Bordeaux \par}
{\scshape\Large Master 1 Informatique  \par}
\vspace{3cm}

{\Huge\bfseries Projet de Programmation\par}
{\Huge\bfseries Le Clicodrome de LEFFF \par}
\vspace{0.5cm}
{\Large\itshape Cahier des besoins\par}
{\large 01 février 2019\par}

\vfill
réalisé par \par
BAKIR \textsc{Fatima Ezzahra} \par
JELLITE \textsc{Oumayma} \par
NEDELEC \textsc{Guillaume} \par
SYLLA  \textsc{Alfred Aboubacar} \par
\vfill

{\large Enseignants responsables : Philippe NARBEL et Vincent PENELLE\par}

\end{titlepage}

\section{Introduction}

\subsection{Présentation du projet}

Le LeFFF est un lexique de formes fléchies du français à large couverture, associant à chaque forme son lemme, ses traits morphologiques et d'autres champs récemment disponibles.
Ce n'est donc pas une simple liste de mots, mais une ressource complexe constituée de :
\begin{itemize}  
  \item Lexème ou grammème (ie Prendre)
  \item Vocables (ie Prendre=saisir, Prendre=recevoir)
  \item Catégories syntaxiques (ie Verbe)
  \item Sous-catégories syntaxiques (ie Transitif, passivable)
  \item Catégories grammaticales (ie Nombre$\rightarrow\{sing, plur\}$, Personne$\rightarrow\{1, 2, 3\}$)
  \item Règles de flexion (ie Table de conjugaison de prendre - Stem=.*(pren|mett))
  \item Valence (ie objet nominal, oblique en à)
  \item Réalisation syntaxique (ie passif en par)
  \item Phraséologie (ie "Prendre ombrage", "Prendre ses jambes à son cou")
  \item Collocation (ie "Prendre une initiative")
  \item Fonctions lexicales (ie Magn: "Prendre une belle initiative")
\end{itemize}

\smallbreak

Ses données dont des structures complexes sur lesquelles opèrent des unificateurs et des transducteurs.
\newline Les transducteurs permettent par composition de construire les formes fléchies. 
Ce sont des automates finis avec sorties permettant de transformer des mots d'un alphabet entrée en un ou plusieurs mot sur un alphabet de sortie.
\newline Les annotations morphosyntaxiques et syntaxiques sont obtenues à partir de l'unification de structures (données du LeFFF).

\smallbreak
Ce lexique souffre depuis sa création du manque de moyens techniques permettant de le modifier et de l'exploiter.
L'objectif du projet est donc de réaliser un site web permettant une interaction des utilisateurs sur le lexique (ajout / modification / suppression de mots et de formes fléchies). 
De plus il faudra réaliser un compilateur permettant de construire des unificateurs et des transducteurs afin de concrétiser ces interactions.

\newpage
\subsection{Description de l'existant et choix de conception}

Pour ce projet, nous avons a notre de disposition le lexique LeFFF mis à jour en avril 2006 au format texte (.txt).
Ce lexique contient un ensemble de forme flechies de mots français présentés de la manière suivante :
\begin{verbatim}
  débouler	v	[pred='débouler_____1<suj:(sn|sinf|scompl),obj:(sn)>',cat=v,@W]
  à bon marché	adv	[pred='à bon marché_____1',cat=adv,advGP=+]
  à bientôt	  pres	[pred='à bientôt_____1',advGP=+]
  Égypte		np	[pred='Égypte_____1<suj:(sn)>',@loc,@fs]
\end{verbatim}

EXPLIQUER LA SYNTAXE DU LEFFF (EX CODE CI DESSUS)

Ce lexique est dsponible sous licence « LGPLLR » (Lesser General Public License For Linguistic Resources) à l'adresse suivante : 
\\ \begin{center}http://www.labri.fr/perso/clement/lefff/telechargement.html\end{center}

  \section{Analyse des besoins}

\subsection{Les besoins fonctionnels:}

Sur le site internet, il existe 4 types d’utilisateurs : Les Administrateurs (A) , les Éditeurs ou linguistes (E), les Modérateurs (M) et les Visiteurs (V).
Chacun de ses utilisateurs possède des accès particuliers sur le site internet.
Les visiteurs ne possèdent pas de compte sur la plate-forme et ne peuvent donc pas s'authentifier . 

\subsubsection{Le site web interactif}
\begin{tabularx}{\textwidth}{|l|c|X|}
  \hline
  \textbf{Besoins} & 
  \textbf{Acteurs} & 
  \textbf{Explication} \\
  \hline
  Authentifier les utilisateurs & 
  A, E, M & 
  Chaque utilisateur ayant un compte sur le site peut s'authentifier pour avoir accès a l'interface d'administration \\ 
  \hline
  Gérer les rôles & A
   & Permet d'assigner les ôles (A,E,M) aux différents utilisateurs ayant un compte sur le site.
 \\
  \hline
  Bannir un utilisateur& 
  A & 
  L'administrateur peut bannir tous les autres utilisateurs de la plateforme \\
  \hline
  Rechercher un mot & 
  A , E, M, V & 
  Rechercher un mot dans le lexique pour accéder à ses données \\
  \hline
  Consulter un mot &
  A, E, M, V &
  Permet de consulter les données d'un mot du lexique \\
  \hline
  Modifier un mot &
  A, E, M & 
  Permet de modifier les données liées à un mot \\
  \hline
  Supprimer un mot &
  A, E, M & 
  Permet de supprimer un mot du lexique \\
  \hline
  Faire un signalement & 
  V &
  Permet de signaler une erreur et de proposer une éventuelle correction \\
  \hline
  Valider inscription Éditeur &
  A &
  Après l'inscription d'un éditeur sur le site, l'administrateur doit activer son compte après vérification \\
  \hline
  Signaler un mot &
  V, E &
  Un visiteur ou un éditeur peut signaler un mot qui comporte des erreurs ou qui est fausse \\
  \hline
  Exporter le lexique &
  V, E, A , M &
  Les utilisateurs du site peuvent exporter LeFFF du site sous différents formats de fichier (a définir)\\
  \hline
\end{tabularx}

\subsubsection{Compilateur PFM}
\begin{tabularx}{\textwidth}{|l|c|X|}
  \hline
  Besoins & Acteurs & Explication \\
  \hline
  Générer le lexique d'un mot &
  Système(PFM)
  & Quand un utilisateur veut consulter le lexique d'un mot via le site web , une requête est transmise à notre Systeme(Back-end) qui lui renvoie le lexique du mot demandé \\
  \hline
  ...
  & ... 
  & ... \\

  \hline
\end{tabularx}
\subsection{Contraintes}
-L'historique : La date de la dernière modification ainsi que l'utilisateur qui a modifié doivent être enregistré.\\
-La gestion de la sécurité est la principale contrainte de notre système.Le site web doit possède une gestion de privilège et de niveaux d'accés pour les différents types d'utilisateurs. Selon leurs statuts, le contenu des pages varie.
\subsection{Les besoins non fonctionnels}

\subsubsection{Sécurité}
\begin{itemize}
 \item La protection des données est faite à l'aide d'une page d'authentification.Il est impossible de modifier ou supprimer des données sans se connecter et avoir les droits nécessaires. 
 \item Déconnexion automatique après une période d'inactivité (durée à définir)
 \item Mise en place de logs afin de tracer les changements effectués.
\end{itemize}
\subsubsection{Ergonomie}
\begin{itemize}
\item Une boite de dialogue ou un message de confirmation sera affiché pour chaque opération effectuée par l'utilisateur.
\item Des couleurs des icônes, des boutons, et des expression significatifs seront implémentée afin que le site soit utilisée avec le maximum de confort et d'efficacité.
\end{itemize}
\subsubsection{Audit}
\begin{itemize}
\item Éléments de vérification dans les champs de saisies.
\end{itemize}

\section{Réparation des tâches et Avancement du projet}

\subsection{Planning du projet}

\begin{tabularx}{\textwidth}{|l|c|X|}
  \hline
  \textbf{Tâche } & 
  \textbf{Date de début} & 
  \textbf{Date de fin} \\
  \hline
  \hline
  \hline
  Capture des besoins fonctionnels :& ../01/2019 & ../01/2019
  \\ 
   \hline
    \hline
     \hline
 
 Étude de projet : & ../01/2019 &  ../01/2019
    \\
  \hline
 Reherche sur Internet sur les outils à utiliser  & ../01/2019 &  ../01/2019
    \\
  \hline
 Définition des besoins et installation des outils  & ../01/2019 &  ../01/2019
    \\
  \hline
 \hline
  \hline
  Analyse et conception :& ../01/2019 & ../01/2019
  \\ 
   \hline
  \hline
   \hline
  Les diagrammes des cas d'utilisation & ../01/2019 &  ../01/2019 
    \\
  \hline
 Description contextuelle & ../01/2019 &  ../01/2019
    \\
  \hline
  Diagrammes de séquence & ../01/2019 &  ../01/2019
    \\
    \hline
     \hline
      \hline
 
Codage et test: & ../01/2019 &  ../01/2019
    \\
      \hline
       \hline
        \hline
    
La création de la base de données & ../01/2019 &  ../01/2019
    \\
     \hline
le codage des entités & ../01/2019 &  ../01/2019
    \\
         \hline
la création des interfaces & ../01/2019 &  ../01/2019
    \\
       \hline
        \hline
         \hline

Rédaction du Rapport: & ../01/2019 &  ../01/2019
  \\
  \hline
   \hline
    \hline

\end{tabularx}

\subsection{Diagramme de Gantt du projet}

\section{Outils de développement}

Ce projet peut être divisé en 3 parties :
\begin{itemize}  
  \item le site web
  \item le compilateur 
  \item la base de données
\end{itemize}

\subsection{Back-end du site web : PHP avec le framework Symfony 4}
\begin{center}
  \includegraphics[width=2cm]{img/php.png}
  \includegraphics[width=2cm]{img/symfony.png}
\end{center}

Nous avons choisi de développer le site web en PHP car c'est un langage permettant de faire de la programmation orienté objet. 
De plus c'est un langage très performant et simple à mettre en place.
Nous avons décidé d'utiliser le framework Symfony car un framework a beaucoup d'avantages comme :
\begin{itemize}  
  \item une architecture MVC (Modèle-Vue-Contrôleur) permettant de découper le code en suivant une logique métier.
  \item la facilité de création de tests automatisés
  \item une sécurisation de l'application web (XSS, CSRF et injection SQL)
  \item Un ensemble de packages déjà développés à intégrer à l'application.
  \item Des performances optimisés avec une utilisation du cache optimisée
  \item Une communauté importante et beaucoup de documentation
\end{itemize}

\smallbreak

Nous utiliserons ce framework dans sa version 4.1 car c'est une version très légère par rapport à la version 3. 
En effet aucun packages complémentaires au noyau n'est installé. De plus le déploiement et la configuration du framework a été très simplifiée dans cette version.
Cette version de Symfony nécessite une version de PHP supérieure ou égale à 7.1.3

\subsection{Front-end du site web : Framework Javascript Angular}
\begin{center}
  \includegraphics[width=2cm]{img/angular.png}
\end{center}
On a décidé d'utiliser un framework Javascript pour la partie front-end du site.
Les framework javascript permettent la création d'interface dynamique tout en gardant un code très structuré, ce qui facilite grandement la maintenance de l'application par la suite.
L'utilisation de Angular avec TypeScript permet de typer le javascript et d'adoper le nouveau standard Javascript ES6.

\subsection{Compilateur PFM : Java}
\begin{center}
  \includegraphics[width=2cm]{img/java.png}
\end{center}
On a choisi d'utiliser le langage Java pour le compilateur PFM car c'est un langage orienté objet. 
Cette caractéristique permet la mise en place du DDD (Domain Driven Design) qui va permettre une structure de l'application autour du domaine.
Ce langage dispose de nombreuses API (Application Programming Interface) publiques pouvant être utilisées sans avoir a réimplémenter les fonctionnalités de ces API. 
De plus, des librairies pour la manipulation de PFM sont disponibles pour le Java, un élément nécessaire pour ce projet.

\subsection{Base de données : MySQL}
\begin{center}
  \includegraphics[width=2cm]{img/mysql.png}
\end{center}
Pour la base de données, le choix s'est porté sur MySQL pour sa facilité d'utilisation et ses bonnes performances. 
De plus, ce Système de Gestion de Base de Données (SGBD) s'intègre facilement à un environnement PHP tel que celui qu'on le souhaite utiliser.

\section{Références}
Lien php / symfony / angular / TypeScript/ ES6 / mysql / java
+ Biblio

https://app.moqups.com/GuillaumeNed33/Tr6FjBELx1/view
 

\section{Questionnement}
\begin{itemize}  
  \item que fait réellement le compilateur ?
  \item Quand doit il être utiliser ?
  \item Que prend il en entrée ?
  \item Quel type de données renvoie t -il?
  \item Le compilateur est il nécessaire si tout est enregistré en base ?
  \item Que signifie le format de données du LeFFF (voir dans l'intro) ?
  \item Comment lier les formes fléchies (pour configurer la base) ?
\end{itemize}
\bibliography{references}
\bibliographystyle{plain}

\end{document}
