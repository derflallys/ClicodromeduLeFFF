\documentclass[12pt,a4paper]{article}

\usepackage{fullpage}
\usepackage[utf8]{inputenc}
\usepackage{amsfonts}
\usepackage{amssymb}
\usepackage[french]{babel}
\usepackage[cyr]{aeguill}
\usepackage{natbib}
\usepackage{graphicx}
\usepackage{tabularx}
\setlength{\parindent}{2em}
\setlength{\parskip}{1em}

\newcommand{\quotes}[1]{``#1''}

\begin{document}

\begin{titlepage}
\centering
{\scshape\LARGE Université de Bordeaux \par}
{\scshape\Large Master 1 Génie Logiciel  \par}
\vspace{3cm}

{\Huge\bfseries PDP -Le Clicodrome de LEFFF \par}
\vspace{0.5cm}
{\Large\itshape Cahier des besoins\par}

\vfill
réalisé par \par
BAKIR \textsc{Fatima Ezzahra} \par
JELLITE \textsc{Oumayma} \par
NEDELEC \textsc{Guillaume} \par
SYLLA  \textsc{Alfred Aboubacar} \par
\vfill

{\large 01 février 2019\par}

\end{titlepage}

\section{Présentation générale }

\subsection{Description de l'existant}

 \subsection*{\textbf{LEFF} }
\smallbreak LEFFF:lexique des formes fléchis du français est une nouvelle technique pour la construction de lexiques morphologiques a large couverture. Elle repose sur l'idée  d'associer a chaque forme une règle,ses traits morphologiques et d'autres champs .
\smallbreak la technique est validé  en extrayant des verbes et des adjectifs sur un corpus français général de 25 millions de mots. Comparé à d'autres ressources lexicales disponibles en français,les résultats sont très satisfaisants, car lefff couvre de nombreux mots, souvent dérivés, qui ne sont pas toujours présents dans d’autres langues.De plus, il est généralisable à toute langue ayant une morphologie substantielle.

\smallbreak Ce lexique souffre depuis sa création du manque de moyens techniques permettant de le modifier et de l'exploiter automatiquement.


 \subsection*{\textbf{PFM}}


\subsection{Objectif}
\smallbreak 
Le but du projet c'est de concevoir un site internet interactif conformément aux moyens existants déjà,qui permet d'accéder au lexique déjà existant LEFF en utilisant le compilateur PFM  pour l'enrichir et l'extraire .



\section{Les besoins fonctionnels:}

Sur le site internet, il existe 4 types d’utilisateurs : Les Administrateurs (A) , les Éditeurs ou linguistes (E), les Modérateurs (M) et les Visiteurs (V).
Chacun de ses utilisateurs possède des accès particuliers sur le site internet.
Les visiteurs ne possèdent pas de compte sur la plate-forme et ne peuvent donc pas s'authentifier . 

\subsection{Le site web interactif}
\begin{tabularx}{\textwidth}{|l|c|X|}
  \hline
  \textbf{Besoins} & 
  \textbf{Acteurs} & 
  \textbf{Explication} \\
  \hline
  Authentifier les utilisateurs & 
  A, E, M & 
  Chaque utilisateur ayant un compte sur le site peut s'authentifier pour avoir accès a l'interface d'administration \\ 
  \hline
  Gérer les rôles & 
  A & 
  Permet d'assigner les rôles (A, E, M) aux différents utilisateurs ayant un compte sur le site \\
  \hline
  Bannir un utilisateur& 
  A & 
  L'administrateur peut bannir tous les autres utilisateurs de la plateforme \\
  \hline
  Rechercher un mot & 
  A , E, M, V & 
  Rechercher un mot dans le lexique pour accéder à ses données \\
  \hline
  Consulter un mot &
  A, E, M, V &
  Permet de consulter les données d'un mot du lexique \\
  \hline
  Modifier un mot &
  A, E, M & 
  Permet de modifier les données liées à un mot \\
  \hline
  Supprimer un mot &
  A, E, M & 
  Permet de supprimer un mot du lexique \\
  \hline
  Faire un signalement & 
  V &
  Permet de signaler une erreur et de proposer une éventuelle correction \\
  \hline
  Valider inscription Éditeur &
  A &
  Après l'inscription d'un éditeur sur le site, l'administrateur doit activer son compte après vérification \\
  \hline
  Signaler un mot &
  V, E &
  Un visiteur ou un éditeur peut signaler un mot qui comporte des erreurs ou qui est fausse \\
  \hline
  Exporter le lexique &
  V, E, A , M &
  Les utilisateurs du site peuvent exporter leFFF du site sous différents formats de fichier (a définir)\\
  \hline
\end{tabularx}

\subsection{Compilateur PFM}
\begin{tabularx}{\textwidth}{|l|c|X|}
  \hline
  Besoins & Acteurs & Explication \\
  \hline
  Générer le lexique d'un mot &
  Système(PFM)
  & Quand un utilisateur veut consulter le lexique d'un mot via le site web , une requête est transmise à notre Systeme(Back-end) qui lui renvoie le lexique du mot demandé \\
  \hline
  ...
  & ... 
  & ... \\

  \hline
\end{tabularx}

\section{Les besoins non fonctionnels}
\subsection{Sécurité}
\begin{itemize}
 \item La protection des données est faite à l'aide d'une page d'authentification.Il est impossible de modifier ou supprimer des données sans se connecter et avoir les droits nécessaires. 
 \item Déconnexion automatique après une période d'inactivité (durée à définir)
 \item Mise en place de logs afin de tracer les changements effectués.
\end{itemize}
\subsection{Ergonomie}
\begin{itemize}
\item Une boite de dialogue ou un message de confirmation sera affiché pour chaque opération effectuée par l'utilisateur.
\item Des couleurs des icônes, des boutons, et des expression significatifs seront implémentée afin que le site soit utilisée avec le maximum de confort et d'efficacité.
\end{itemize}
\subsection{Audit}
\begin{itemize}
\item Éléments de vérification dans les champs de saisies.
\end{itemize}

\end{document}
