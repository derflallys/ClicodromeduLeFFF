\documentclass{article}
\usepackage[utf8]{inputenc}

\title{PdP - Le Clicodrome de LeFFF \\ Compte rendu du TD6}
\author{}
\date{Mercredi 27 Mars 2019}

\begin{document}

\maketitle

\section{Contexte}
Cette séance de TD avait pour but de faire un point sur l'avancement du projet au niveau du code et du mémoire.

\section{Partie Code}
\begin{itemize}
    \item Forcer le niveau d'application 0 quand on redefini le radical
    \item Parsers à mettre en tant que prototypes
    \item Background de l'appli en blanc
    \item Export oubli d'un retour à la ligne
    \item Import -> Calcul des performances
    \item Formes fléchies : afficher les tags qui la génère
    \item Génération des formes fléchies : rendre le comportement des tagWord similaire à tagCategory
\end{itemize}

\section{Le mémoire}
\begin{itemize}
    \item Expliquer la dissociation entre tagWord et tagCategory dans la règle PFM -> changement du formalisme (le fait que ca nous permet d'optimiser nos algorithmes ect...)
    \item Attention fautes d'orthographes / grammaire
    \item Ne pas être trop flou (ex : "interpreteur" -> de quoi ? pour quoi ?...)
    \item Les explications doivent être compréhensible de tous
    \item Evitez les formulations "il faut savoir que" / "y'a des boutons qui font ca"
    \item Montrer que notre implémentation est transposable à n'importe quel autre langage
    \item Nom de famille pas en majuscule
    \item Persister -> Enregistrer
    \item Export : Expliquer pourquoi la catégorie forme fléchie et pas la catégorie du lemme
    \item Sécurité : baiiser la priorité et séparer en plusieurs sous-besoins
    \item Performances : facteur limitant -> temps de réponse de la BD (aucun controle dessus)
    \item Performances : calculé la performance de l'application de règle pour une combinaison de tags
    \item Ordre de priorité -> Niveau d'application de la règle
    \item Confit de règle (règle qui prend a;b et une autre b;c -> Amélioration possible : renvoyer une erreur à l'utilisateur)
    \item Citer références
    \item Choix MySQL : Perf raisonnable ? Facile d'utilsiation, Demandé par le client
    \item Dans la partie Architecture du site : ne parler que de ce qui a été implémenté !
\end{itemize}

\end{document}
