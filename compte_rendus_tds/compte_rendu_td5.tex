\documentclass{article}
\usepackage[utf8]{inputenc}

\title{PdP - Le Clicodrome de LeFFF \\ Compte rendu du TD5}
\author{}
\date{Mercredi 06 Mars 2019}

\begin{document}

\maketitle

\section{Contexte}
Cette séance de TD avait pour but de 
faire un débriefing de la 1ère release, de parler du mémoire et de ce que l'on compte faire pour la suite.

\section{Le mémoire}
Le mémoire doit sebaser sur le cahier des besoins. Il sera composé de :
\begin{itemize}
    \item Une introduction du projet
    \item Une analyse de besoins
    \item De l'architecture du projet (Algo / Choix techniques...)
    \item Présentation des tests
    \item Une conclusion
\end{itemize}
Attntion à ne pas trop détaillé le fonctionnement des framework mais bien montrer ce que nous avons réalisé.
Présenter seulement les aspects utiles du framework pour le projet.

\section{Objectifs}
Les objectifs principaux sont :- 
\begin{itemize}
    \item Créer un interpréteur des règles PFM
    \item Faire la liaison mot/règles (Savoir quels règles appliquer à un mot)
    \item Généricité de l'application (doit pouvoir s'adapter à un lexique quelqconque)
\end{itemize}

Ne pas hésiter à commiter des prototypes.
Implémenter les examples en perses pour nous servir d'exemples.

Faire a liaisons MOT => REGLES de manières générique ?
\end{document}
