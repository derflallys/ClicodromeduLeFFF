\documentclass{article}
\usepackage[utf8]{inputenc}

\title{PdP - Le Clicodrome de LeFFF \\ Comtpe rendu du TD3}
\author{}
\date{Mercredi 06 Février 2019}

\begin{document}

\maketitle

\section{Contexte}
Cette séance de TD avait pour but d'avoir le retour de notre chargé de TD sur le cahier des besoins que l'on a rendu et d'établir ce que nous voulions faire pour la première Realease de code du 15/02/2019.

\section{Retour du cahier des besoins}
\smallbreak Le cahier a été beacoup mieux développé qu'à la dernière version (TD2)
Petit bémol sur la partie des besoins où "le coeur" du projet, le compilateur PFM n'est pas assez développé.
En parallèle, les autres besoins sont très développé, ce qui marque encore plus le manque de détails de la partie compilateur.
Il manque encore une partie "faisabilité" sur les besoins. (Qu'est ce qui est faisable ? Comment on pense le faire ? Comme on l'a fait ?)


\section{Objectif de la 1ère realease}
Voici les tâches que l'on souhaite remplir pour la première realease dans l'ordre de priorité : 
\begin{itemize}
    \item Création de la base de données (Architecture)
    \item Parseur du lefff (format txt et/ou mlex)
    \item Alimentation de la base de données avec le lefff
    \item Documentation des outils
    \item Début du développement de l'interface web (front)
\end{itemize}

\section{Préparation de l'audit}
\smallbreak Bien se renseigner sur le compilateur et son fonctionnement afin de pouvoir expliquer simplement comment le programme marche surant l'audit.
Il faut vraiment que l'on maitrise parfaitement ce sujet, qu'ils soit développé ou non.
\begin{itemize}
    \item PFM, c'est quoi ?
    \item Lister les règles du PFM ( dans un fichier passé en entrée de notre programme ?)
    \item Comment marche l'algo ?
    \item Existe t il des éléments existants ?
    \item Est-on obliger de développé un outil externe ou cela s'intègre dans le back du site (en PHP?)
\end{itemize}
\end{document}
