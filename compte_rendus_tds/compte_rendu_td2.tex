\documentclass{article}
\usepackage[utf8]{inputenc}

\title{PdP - Le Clicodrome de LeFFF \\ Comtpe rendu du TD2}
\author{}
\date{Mercredi 30 Janvier 2019}

\begin{document}

\maketitle

\section{Contexte}
Cette séance de TD avait pour but de faire un point sur la 2ème version du cahier des besoins et de répondre à nos questionnement sur le compilateur PFM.

\section{Remarques et conseils}
\subsection{Partie d'introduction}
\begin{itemize}
    \item Introduction imcompréhensible, il faut citer les références quand on reprend des explications d'une source externe
    \item Citer les références du fichier .bibtex dans le cahier des besoins pour quelles apparaissent.
    \item Bien expliquer le projet, de quoi on part ? les problèmes de l'existant ? Notre solution proposée...
    \item Transducteurs/unificateurs $\longrightarrow$ quel est le rapport ? c'est quoi ? ca fait quoi ?
    \item Définir le leFFF, les formes fléchies avec des infos compréhensible de tous
    \item Bien parler le l'existant ? Comment on le manipule ect...
    \item Objectif du projet a bien détaillé
\end{itemize}

\subsection{Partie besoin}
\begin{itemize}
    \item Transformer le lexique (.txt) en une base de données $\longrightarrow$ Besoin principal
    \item Les besoins non fonctionnels qu'on a rensigné sont pour la majorité des besoins fonctionnels
    \item Partie sécurité incomplete $\longrightarrow$ Protection de la base, Injection SQL, faille XSS ect...
\end{itemize}

\subsection{Partie Planning}
\begin{itemize}
    \item Reformuler les taches
    \item Certaines taches trop vagues (ex: codage)
    \item Certaines taches trop détaillés
\end{itemize}

\subsection{Partie Outils de dev}
\begin{itemize}
    \item Logo inutile
    \item Indiquer dans les raisons de ces choix, des raisons en lien avec le projet
    \item DDD $\longrightarrow$ expliquer $\longrightarrow$  pourquoi ca serait utile au projet (si ça l'est !)
\end{itemize}

\subsection{Partie projet}
\begin{itemize}
    \item En base, on enregistre seulement le mot et son lème (ses attributs)
    \item Aucune forme fléchies en base, le compilateur peut les générer
    \item Des compilateurs PFM existent, à nous d'exploiter ce qui existe déjà
    \item Notre projet cible le leFFF mais il devraient etre générique à n'importe quel lexique
    \item Forme fléchies du francais : la conjugaison du mot et ses déclinaisons (ex : "mien" est une déclinaison de "moi")
    \item Fonctionnalité d'export du lexique (avec ou sans les formes fléchies) $\longrightarrow$ format correspondant au format d'entrée (à la base du projet : .txt)
    \item Importer le lexique en base
\end{itemize}
\end{document}
