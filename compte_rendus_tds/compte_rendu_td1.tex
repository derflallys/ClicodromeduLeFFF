\documentclass{article}
\usepackage[utf8]{inputenc}

\title{PdP - Le Clicodrome de LeFFF \\ Comtpe rendu du TD1}
\author{}
\date{Mercredi 23 Janvier 2019}

\begin{document}

\maketitle

\section{Contexte}
Cette séance de TD avait pour but d'analyser notre première version du besoin avec notre chargé de TD, Vincent Penelle. Il fallait donc voir si nous avions bien compris le besoin du client et si nous l'avons correctement retranscrit dans le document.
L'objectif étant de rendre une version finalisée du cahier des besoins pour le vendredi 1er février.


\section{Remarques positives}
\begin{itemize}
    \item Bon découpage des besoins utilisateurs pour la partie concernant le site web
    \item Explication des rôles 
    \item Partie "Sécurité" pertinente
\end{itemize}

\section{Remarques négatives et conseils}
\begin{itemize}
    \item Absence d'introduction présentant le projet. (Présenter le projet, ce que l'on compte faire, comment et pourquoi.)
    \item Absence de bibliographie
    \item Absence de digramme de Gantt
    \item Parties "Audit" et "Ergonomie" pas claires
    \item Besoins utilisateurs bien identifiés mais les autres besoins non.
    \item Besoins pas détaillés (Qu'est ce qu'il se passe techniquement)
    \item Définition des outils manquant
    \item La problèmatique du compilateur PFM pas comprise et donc manquante
    \item Développer les besoins non fonctionnels
    \item Le coeur du projet, c'est le compilateur
    \item Préciser les choix des technologies et pourquoi
    \item Utilisation de maquette et de diagrammes (de manière non abusive) peut être pertinent
    \item Quels sont les éléments existants, les elements à developper ? Comment va t'on utiliser ces éléments
    \item Se renseigner sur les données du LeFFF
\end{itemize}

\end{document}
