\documentclass{article}
\usepackage[utf8]{inputenc}

\title{PdP - Le Clicodrome de LeFFF \\ Compte rendu du TD4}
\author{}
\date{Mercredi 13 Février 2019}

\begin{document}

\maketitle

\section{Contexte}
Cette séance de TD avait pour but de montrer à notre chargé de TD l'état actuel de notre code (qu'est ce qu'on a codé exactement): les interfaces Web qu'on a commencé à faire et le déploiement de la base de données. 

\section{Remarques: Code}
\begin{itemize}
    \item Interaction base de données et interfaces graphiques.
    \item les fichiers README a bien détaillé: Décrire tous les fichiers qu'on a créé.
    \item Documentation du code.
\end{itemize}

\section{Conseils: PFM}
Voici les étapes que l'on souhaite suivre pour générer les formes fléchies:
\begin{itemize}
    \item Trouver/Chercher les règles générales du PFM, et comme proposition ces règles peuvent être stocker sur une base de données.
    \item Créer des fonctions qui permettent d'appliquer ces règles sur les lemmes pour retourner la forme fléchie de chaque mot.
\end{itemize}

\section{Préparation de l'audit}
\smallbreak Bien se renseigner sur l'objectif principale du projet et et bien comprendre le fonctionnement du programme qu'on souhaite créer afin de pouvoir expliquer simplement comment il marche durant l'audit.
Il faut vraiment que l'on maîtrise parfaitement ce sujet, qu'ils soit développé ou non.
\begin{itemize}
    \item PFM, c'est quoi ?
    \item Lister les règles du PFM ( dans un fichier passé en entrée de notre programme ?)
  
\end{itemize}
\end{document}
